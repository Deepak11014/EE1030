%iffalse
\let\negmedspace\undefined
\let\negthickspace\undefined
\documentclass[journal,12pt,twocolumn]{IEEEtran}
\usepackage{cite}
\usepackage{amsmath,amssymb,amsfonts,amsthm}
\usepackage{algorithmic}
\usepackage{graphicx}
\usepackage{textcomp}
\usepackage{xcolor}
\usepackage{txfonts}
\usepackage{listings}
\usepackage{enumitem}
\usepackage{mathtools}
\usepackage{gensymb}
\usepackage{comment}
\usepackage[breaklinks=true]{hyperref}
\usepackage{tkz-euclide} 
\usepackage{listings}
\usepackage{gvv}                                        
%\def\inputGnumericTable{}                                 
\usepackage[latin1]{inputenc}                                
\usepackage{color}                                            
\usepackage{array}                                            
\usepackage{longtable}                                       
\usepackage{calc}                                             
\usepackage{multirow}                                         
\usepackage{hhline}                                           
\usepackage{ifthen}                                           
\usepackage{lscape}
\usepackage{tabularx}
\usepackage{array}
\usepackage{float}


\newtheorem{theorem}{Theorem}[section]
\newtheorem{problem}{Problem}
\newtheorem{proposition}{Proposition}[section]
\newtheorem{lemma}{Lemma}[section]
\newtheorem{corollary}[theorem]{Corollary}
\newtheorem{example}{Example}[section]
\newtheorem{definition}[problem]{Definition}
\newcommand{\BEQA}{\begin{eqnarray}}
\newcommand{\EEQA}{\end{eqnarray}}
\newcommand{\define}{\stackrel{\triangle}{=}}
\theoremstyle{remark}
\newtheorem{rem}{Remark}

% Marks the beginning of the document
\begin{document}
\bibliographystyle{IEEEtran}
\vspace{3cm}

\title{Chapter 6 \\Sequence and Series}
\author{EE24BTECH11014-Deepak}
\maketitle
\newpage
\bigskip

\renewcommand{\thefigure}{\theenumi}
\renewcommand{\thetable}{\theenumi}
\begin{enumerate}
    \item Sum of the first $ n$ terms of the series \\
$ \frac{1}{2}+ \frac{3}{4}+ \frac{7}{8}+ \frac{15}{16}+ \dots $ is equal to\hfill (1998-2 Marks)
\begin{enumerate}
    \item $2^n-n-1$  
    \item $1-2^{-n}$
    \item $n+2^{-n}-1$
    \item $2^n+1$
    \end{enumerate}
\item The number ${log_2}7$ is \hfill(1990-2 Marks)
    \begin{enumerate}
        \item an integer
        \item a rational number
        \item an irrational number
        \item a prime number
    \end{enumerate}
\item If $ln(a+c),ln(a-c),ln(a-2b+c)$ are in A.P.,then \hfill (1994)
    \begin{enumerate}
        \item $a,b,c$ are in A.P.
        \item $a^2,b^2,c^2$ are in A.P.
        \item $a,b,c$ are in G.P.
        \item $a,b,c$ are in H.P.
    \end{enumerate}
\item Let ${a_1,a_2,\dots a_{10}}$ be in A,P, and ${h_1,h_2, \dots h_{10}}$ be in H.P. If ${a_1}={h_1}=2$ and ${a_{10}}={h_{10}}=3$, then ${a_4h_7}$ is \hfill(1992 - 2 Marks)
    \begin{enumerate}
        \item 2
        \item 3
        \item 5
        \item 6
        \end{enumerate}
\item The harmonic mean of the roots of the equation
        $(5+\sqrt{2})x^2-(4+\sqrt{5})x+8+2\sqrt{5}=0$ is 
        \hfill(1999 - 2 Marks)
        \begin{enumerate}
            \item 2
            \item 4
            \item 6
            \item 8
        \end{enumerate}
\item Consider an infinite geometric series with first term $a$ and common ratio $r$. If its sum is 4 and the second term is 3/4, then \hfill (2000S)
        \begin{enumerate}
            \item $a=4/7,r=3/7$
            \item $a=2,r=3/8$
            \item $a=3/2,r=1/2$
            \item $a=3,r=1/4$
            \end{enumerate}
\item Let $\alpha$,$\beta$ be the roots of $x^2-x+p=0$ and $\gamma$,$\delta$ be the roots of $x^2-4x+q=0$.If $\alpha$,$\beta$,$\gamma$,$\delta$ are in G.P.,then the integral values of of $p$ and $q$ respectively are \hfill(2001S)
            \begin{enumerate}
                \item $-2,-32$
                \item $-2,3$
                \item $-6,3$
                \item $6,-32$
        
    \end{enumerate}
\item Let the positive numbers $a,b,c,d$ be in A.P. Then $abc,abd,acd,bcd$ are \hfill(2001S)
    \begin{enumerate}
        \item NOT in A.P./G.P./H.P
        \item in A.P.
        \item in G.P.
        \item  in H.P.
        \end{enumerate}
\item If the sum of the first $2n$ terms of the A.P. $2,5,8,\dots,$ is equal to the sum of the first $n$ terms of the A.P.$57,59,61,\dots,$ then $n$ equals \hfill(2001S)
        \begin{enumerate}
            \item 10
            \item 12
            \item 11
            \item 13
            \end{enumerate}
\item Suppose $a,b,c$ are in A.P. and $a^2,b^2,c^2$ are in G.P. if $a<b<c$ and $a+b+c=3/2$, then the value of $a$ is \hfill(2002S)
            \begin{enumerate}
             \item $\frac{1}{2\sqrt{2}}$
             \item $\frac{1}{2\sqrt{3}}$
             \item $\frac{1}{2}-\frac{1}{\sqrt{3}}$
             \item $\frac{1}{2}-\frac{1}{\sqrt{2}}$
            \end{enumerate}
\item An infinite G.P. has first term $'x'$ and sum $'5'$ then $x$ belongs to \hfill(2004S)
            \begin{enumerate}
                \item $x<-10$
                \item $-10<x<0$
                \item $0<x<10$
                \item $x>10$
                \end{enumerate}
\item .In the quadratic equation $ax^2+bx+c=0,\triangle=b^2-4ac$ and $\alpha+\beta,\alpha^2+\beta^2,\alpha^3+\beta^3$ are in G.P.where $\alpha,\beta $ are root of $ax^2+bx+c=0$,then \hfill(2005S)
                \begin{enumerate}
                    \item $\triangle\neq0$
                    \item $b\triangle=0$
                    \item $c\triangle=0$
                    \item $\triangle=0$
                \end{enumerate}
\item In the sum of first $n$ terms of an A.P. is $cn^2$, then the sum of squares of these $n$ terms is \hfill(2009)
                \begin{enumerate}
                    \item $\frac{n(4n^2-1)c^2}{6}$
                    \item $\frac{n(4n^2+1)c^2}{3}$
                    \item $\frac{n(4n^2-1)c^2}{3}$
                    \item $\frac{n(4n^2+1)c^2}{6}$
		\end{enumerate}              \item Let ${a_1,a_2,a_3,\dots}$ be in harmonic progression with ${a_1}=5$ and ${a_{20}}=25$. The least positive integer $n$ for which ${a_n<0}$ is \hfill(2012)
                \begin{enumerate}
                    \item 22
                    \item 23
                    \item 24
                    \item 25
                    \end{enumerate}
\item Let ${b_i}>1$ for $i=1,2,\dots,101$. Suppose ${log_e}{b_1},{log_e}{b_2},\dots,{log_e}{b_{101}}$ are in Arithmetic Progression (A.P.) with the common difference ${log_e}2$. Suppose ${a_1,a_2,\dots,a_{101}}$ are in A.P. such that ${a_1=b_1}$ and ${a_{51}=b_{51}}$. If $t={b_1+b_2+\dots+b_{51}}$ and $s={a_1+a_2+\dots+a_{53}}$, then \hfill (JEE Adv. 2016)
                    \begin{enumerate}
                        \item $s>t$ and ${a_{101}>b_{101}}$
                        \item $s>t$ and ${a_{101}<b_{101}}$
                        \item $s<t$ and ${a_{101}>b_{101}}$
                        \item $s<t$ and ${a_{101}<b_{101}}$
                        \end{enumerate}
                        
    
\end{enumerate}
\end{document}
